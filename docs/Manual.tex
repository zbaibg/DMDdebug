\documentclass{article}

\usepackage[utf8]{inputenc}
\usepackage{amsmath} 
\usepackage{amssymb} 
\usepackage{amsthm}
\usepackage{bm} % allows use of bold greek letters in math mode using the \bm{...} command
\usepackage{empheq}
\usepackage{float} % improves the interface for defining floating objects such as figures and tables
\usepackage[textwidth=15cm]{geometry} % for easy management of document margins and the document page size
\usepackage{mathtools} % supplements amsmath, some additional functionality, some bugs fixed
\usepackage{mhchem} % allows you to easily type chemical species and equations
\usepackage[numbers]{natbib} % gives additional citation options and styles, often used for journal submission
\usepackage{physics}
\usepackage{siunitx} % helps type SI units correctly
\usepackage{tensor}
\usepackage{textcomp} % provides extra symbols, e.g. arrows like \textrightarrow
\usepackage[hyphens]{url}
\usepackage{xcolor}
\usepackage{xr-hyper}
\usepackage{comment}
\usepackage{booktabs,xltabular}
\usepackage[final]{pdfpages} % allow to insert PDF

\usepackage{hyperref} % gives LaTeX the possibility to manage links within the document or to any URL
\hypersetup{
	colorlinks=true,
	urlcolor=blue,
	linkcolor=blue
}
\usepackage{graphicx} % allows you to insert graphic files within a document
\usepackage{subcaption} % allows to define multiple loats (figures, tables) within one environment giving idividual captions and labels in the form 1a, 1b
\usepackage{listings}

\lstdefinestyle{mystyle}{
	showtabs=false,                  
	tabsize=2
}





\lstset{style=mystyle}
\newcommand\aug{\fboxsep=-\fboxrule\!\!\!\fbox{\strut}\!\!\!}

\title{Density Matrix Dynamics \\
User Manual}
\author{Junqing Xu, Kejun Li, Yuan Ping}
\date{\today}

\begin{document}
	
	\maketitle
	\pagebreak
	\tableofcontents
	
	\pagebreak
	
\section{Introduction}
DMD is a first-principles real time density matrix dynamics code. It uses wannierized ground state electronic structure calculated at the density funcional theory (DFT) level along with various scattering interactions and coherent contributions. Currently, DMD only interfaces with JDFTx, a general purpose DFT code and relies on an intermediate brigding code, FeynWann, for inialization and intermemdiate calculations of the scattering processes. 

\section{Installation}

Create build directory at the same level as \verb|src| directory in the main DMD directory. Change  \verb|make.inc| to point to dependencies according to your specific system ennvironment.
\begin{enumerate}
\item Make sure intel mpi compiler and command \verb|mpiicpc| or its equivalent exist.
\item  GSL and MKL must be installed and ensure \verb|MKLROOT|  points to correct location.
\item Modify \verb|GRL_DIR| in \verb|make.inc| and \verb|SRC_DIRS| in \verb|Makefile|, if necessary.
\item If there is no "build" directory, make it.
\item Issue \verb|make| command in the main directory. 
\end{enumerate}

	
\section{DMD Input}%%%%%%%%
DMD parses input commands from a simple text file with one command per line, first entry is the command key and the second is command value, if any. 
\subsection{DMD Commands}
	\begin{xltabular}{\textwidth}{X X X}
		\caption{Parameters in DMD input param.in for task control and algorithms.}\\
		\toprule\toprule
		parameter & input & meaning \\
		\midrule
		
		restart & 1 (default) or 0 & If 1, start from scratch; else if 0, restart.\\
		\midrule
		
		alg\_scatt & lindblad (default) or conventional & It determines the form of the scattering term of the master equation.\\
		\midrule
		
		alg\_eph\_sepr\_eh, alg\_eph\_need\_elec, alg\_eph\_need\_hole & 1 or 0 & If alg\_eph\_sepr\_eh = 0, electron and hole will be calculated together. If alg\_eph\_sepr\_eh = 1, electron and hole will be calculated separately. Then, if ``ePhOnlyElec 1'' is in lindbladInit.in, set  alg\_eph\_need\_elec = 1 and alg\_eph\_need\_hole = 0. If ``ePhOnlyHole 1'' is in lindbladInit.in, set  alg\_eph\_need\_elec = 0 and alg\_eph\_need\_hole = 1.\\
		\midrule
		
		alg\_sparseP & 1 or 0 (default) & If 1, the code will convert the generalized scattering-rate matrix P to a sparse matrix. If 0, matrix P is kept dense. Search for ``ns\_tot'' in output to see how many elements of $\mathrm{P}_i$ matrices are larger than several thresholds (default 1e-40) generally, larger energy range and smaller smearing make $\mathrm{P}_i$ more sparse.\\
		\midrule
		
		alg\_phenom\_relax & 1 or 0 (default) & If 1, phenomenon relaxation $\dot{\rho} = -(\rho - \rho_\text{eq}) / \tau_\text{phenom}$ will be turned on. If 0, it will be turned off.\\
		\midrule

		alg\_phenom\_recomb & bool (default 0) & If 1, phenomenonlogical recombination to represent spontaneous emission will be turned on.\\
		
		tau\_phenom\_recomb & float (default $10^{15}$ ps) & Phenomenonlogical recombination time in pico seconds.\\
		\midrule
		
		Bxpert, Bypert, Bzpert & a number, e.g. 0.0 (default) & Magnetic field perturbation, one of the mothods that perturbs a system to generate spin unbalance, in the unit of Tesla. x, y and z are cartesian coordinates consistent to input cell parameters.\\
		\midrule
		
		pumpMode & coherent or lindblad or perturb & Perturbation by pumping, the other method that is used to perturb a system to generate spin unbalance. The pump pulse is Gaussian. coherent : real-time pump using coherent formula $-i[\mathrm{P}(t),\rho]$. lindblad : real-time pump using lindblad formula. perturb : generate an initial pump perturbation.\\
		\midrule
		
		pumpA0 & a number, e.g. 0.0 & Pump amplitude, pump power = (pumpE * pumpA0)$^2$ / ($8\pi \alpha$)\\
		\midrule
		
		pumpPoltype & LC or RC or Ex or Ey & Pump polarization type.\\
		\midrule
		
		pumpE & a number, e.g. 0.1 & Pump energy in eV.\\
		\midrule
		
		pumpTau & a number, e.g. 50 & Pump pulse width in the unit of fs. This introduces a weight function $\mathrm{exp}(-t^2/2\tau^2) / \sqrt{\sqrt{\pi} * \tau}$ into pump amplitude\\
		\midrule
		
		probePoltype1 & LC or RC or Ex or Ey & The first probe polarization type.\\
		\midrule
		
		probePoltype$i$ & LC or RC or Ex or Ey & The $i$-th probe polarization type.\\
		\midrule
		
		probeEmin, probeEmax & a number & These parameters specify the probe energy range in the unit of eV.\\
		\midrule
		
		probeDE & a number, e.g. 0.01 & Probe energy step in the unit of eV.\\
		\midrule
		
		probeTau & a number, e.g. 2000 & Probe pulse width in unit the unit of fs.\\
		\midrule\midrule
		
		t0 & a number, e.g. 0 (default) & The initial time of spin dynamics in the unit of fs.\\
		\midrule
		
		tend & a number, e.g. 1e6 & The end time of spin dynamics for reporting in the unit of fs. This number needs to be larger at lower temperature because spin dynamics is slower at lower temperature. 1e6 is used for T=4K.\\
		\midrule
		
		tstep & a number, e.g. 1e3 & Time step in the unit of fs.\\
		\midrule
		
		tstep\_pump & a float & Time step for reporting during pump (pump time center $\pm$6*pumpTau), in the unit of fs.\\
		\midrule
		
		freq\_measure\_ene & a integer & This parameter specifies how often to print energy-resolved observables.\\
		\midrule
		
		temperature & a number, e.g. 300 & Temperature at which calculation is done. Currently, it is actually directly read from the output of the initialization calculation, so that it is not necessary to set temperature.\\
		\midrule
		
		mu & a number, e.g. 0.5 & Chemical potential in the unit of eV, usually relative to VBM, must be in range [dmuMin, dmuMax] in initialization\\
		\midrule
		
		carrier\_density & a number, e.g. 1e18 & Carrier density in the unit of cm$^{-d}$, where d (3,2,1,0) is the dimension of the material. Positive means electron density and negative means hole density. It can be referred to and smaller than the carrier density in the lindblad initialization output lindbladInit.out. When this parameter is used, mu is ignored.\\
		\midrule
		
		tau\_phenom & float & Phenomenon relaxation time in the unit of ps.\\
		\midrule
		
		scrMode & medium or none (default) & "none" means no screening will be computed. "medium" means the dielectric function = interband dielectric constant * intraband dielectric function. interband one must be provided by epsilon\_background. intraband one will be computed by the program.\\
		\midrule
		
		epsilon\_background & a number, e.g. 30 & static dielectric constant $\varepsilon_0$.\\
		\midrule
		
		eeMode & Pee\_update or Pee\_fixed\_at\_eq & The electron-electron scattering matrix P depends on the density matrix $\rho$. There are options of updating the scattering matrix $\mathrm{P^{e-e}}$ (Pee\_update) and fixing the matrix (Pee\_fixed\_at\_eq) during spin dynamics.\\
		\midrule
		
		freq\_update\_ee\_model & integer & This parameter controls how often $\mathrm{P^{e-e}}$ is updating duing spin dynamics. If $<=$ 0, $\mathrm{P^{e-e}}$ is updated every step. \\
		\midrule
		
		alg\_eph\_enable & 1 (default) or 0 & If 1, enable the electron-phonon scattering. If 0, diable electron-phonon scattering.\\
		\midrule
		
		alg\_only\_eimp & 1 or 0 (default) & If 1, only consider the electron-impurity scattering. If 0, consider the scattering processes other than electron-impurity.\\
		\midrule
		
		impMode & ab\_neutral or model\_ionized & Currently, the electron-impurity scattering by neutral defects is computed by supercell method at DFT, and that by charge defects is computed by a charge defect model. If ab\_neutral, electron-impurity scattering considers short-range interaction due to neutral defects. If model\_ionized, the scattering considers long-range interaction due to charge defects.\\
		\midrule
		
		impurity\_density & a number, e.g. 1e18 & Impurity density in the unit of cm$^{-d}$, where d (3,2,1,0) is the dimension of the material.\\
		\midrule
		
		\bottomrule\bottomrule
		\label{tab:dmd}
	\end{xltabular}
	
	

\end{document}
